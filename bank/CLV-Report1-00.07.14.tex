\documentclass{article}

\title{Customer Lifetime Value}
\author{Fateme Fazeli-Asl \\
Mohaddese Mohammadi}


\begin{document}
\maketitle


\begin{abstract}
Customer Lifetime Value is  one of the important indexes for ranking organization customers.
\end{abstract}

\section{Introduction}\label{sec.Intro}

\section{Literature review}\label{sec.review}

\section{Research Methology}\label{sec.Model}
\\ As reviewd in subject literature the relation between customer lifetime value and customer loyalty hasn't investigate through many previose researchs. Naowadays companies look for every way to decrease their costs while increasing profits. One can bring down extra costs is analyse the money waste place during all activites specially marketing and advertisments. Companies and researchers act against them using segmentation customers based on various criteria such as profitability, frequency, lifestyle, loyalty, customer lifetime value and etc. They more just paid attention to one of existed criteria and take marketing strategic based on achived results. Actually maybe if one criteria has positive effect on decreasing marketing cost doesn't have the same effect combining with another criteria. During the present paper we will predict customer loyalty and customer lifetime value then outcomings will compare with together in according with suggested framework.
The main questions of the paper is as following:\\
What does the CLV predict for future.\\
Whether customer with high CLV is identified as loyal customer or not?\\
Can type of product can affect on CLV and loyalty?\\
\\predicting customer lifetime vale:
The famouse formulation of CLV calculating is net present value of earned cash flows of customers during their life in a company. It's easy to calculate CLV just for present but it's not for future. Therefore this paper concentrate on predict CLV in the future using  data mining techniqes. 
\\costomer loyalty:
\\compare loyalty and customer lifetime value:
Maybe most pepole belive  that CLV have determiner role positive on customer loyalty. In the mining that, customers with high CLV score are usually loyal. How true this sentence is? We will answer it during the paper. 
\section{Implementing methology/Experimental example}\label{sec.Example}

\section{Conclusion}\label{sec.Conclusion}

\section*{acknowledge}


\end{document}